\documentclass[12pt, a4paper, twocolumn]{report}
\usepackage[utf8]{inputenc}
\usepackage[russian]{babel}
\usepackage{hyperref}
\usepackage[]{graphicx}

%\makeindex

\newcommand{\PROGNAME}{assembly007}

\input{../../commons/doc/style.tex}
\newcommand{\IMPORTANT}{{\bf ВНИМАНИЕ:~}}

\newcommand{\CTL}[1]{<<{\bf #1}>>}

\newcommand{\CMD}[1]{<<{\tt #1}>>}

\newcommand{\PARAM}[1]{\item {\bf #1} }

\newcommand{\PARAMSECTION}[1]{\vbox{}{\bf Раздел <<#1>>}}


\title{Установка №~1. \\ Регистрация деформации и напряжение в~зависимости от~температуры. \\ Руководство пользователя}
\author{Накин~А.~В.}
\date{Версия программы: 1.0.0\\Ревизия документа: №~1 от \mbox{2 мая 2015~г.}}

\begin{document}

\maketitle

\tableofcontents

\chapter{Общие сведения}

Установка №~1 (далее~--- Установка) представляет собой программно-аппаратный комплекс для регистрации деформации и напряжения упругих материалов (далее~--- Образцов) в зависимости от температуры.

Установка производит съём показаний измерительных приборов, вычисление значений деформации и напряжения со всеми сопутствующими погрешностями и запись результатов в файлы данных.

Управление температурой Образца производится внешними устройствами. Установка сама никак не влияет на температуру Образца, а только регистрирует её.

Одновременно Установка способна работать только с одним единственным Образцом.

\section{Состав Установки}

\subsection{Состав аппаратной части}

Аппаратная часть Установки состоит из персональной ЭВМ (далее~--- ПЭВМ) и следующих приборов:

\begin{itemize}

\item Преобразователь интерфейсов USB/RS-485 АС-4 компании ОВЕН (1~шт.). Предназначен для сопряжения устройств, работающих на шине RS-485, с  ПЭВМ посредством USB интерфейса. Другими словами, АС-4 является контроллером шины RS-485, и к нему подключаются устройства, имеющие данный интерфейс, а сам АС-4 подключается к ПЭВМ.

\item Измеритель-регулятор одноканальный ТРМ-201 компании ОВЕН (1~шт.). Предназначен для определения температуры Образца посредством термопары. Управляется ПЭВМ посредством АС-4.

\end{itemize}

\subsection{Состав программной части}
\label{sec_software}

Программная часть Установки состоит из следующих компонентов.

\begin{itemize}

\item Программа Установки (далее~--- Программа), состоящая из нескольких модулей, написанных на языке Tcl.

\item Интерпретатор языка Tcl и необходимые библиотеки. Дополнительные сведения см. в соответствующих документах.

\item Библиотека, предоставляющая программный интерфейс VISA для доступа к приборам 34410A и E3645A. Например: программный пакет IO Libraries Suite компании Agilent. Дополнительные сведения см. в соответствующих документах.

\item Драйвера устройств, подключаемых непосредственно к ПК.


\end{itemize}

\section{Принцип работы}

Установка периодически с заданной частотой измеряет углы деформации и температуру Образца, вычисляет деформацию и напряжения и записывает результаты в файлы, а также выводит на экран ПЭВМ для оперативного контроля. Оператор вручную управляет температурой Образца.

\label{sec_registration_types}

Частота регистрации (записи в файл) измерений вводится оператором перед началом измерений и может задаваться одним из следующих способов:

\begin{itemize}
\item {\bf Временная зависимость}~--- показания регистрируются с фиксированным временным интервалом, например один раз в секунду.
\item {\bf Температурная зависимость}~--- показания регистрируются с фиксированным температурным шагом, например $1$~Кельвин.
\item \label{sec_reg_type_manual} {\bf Вручную}~--- показания регистрируются по команде оператора.
\end{itemize}

\chapter{Подготовка к работе}

\section{Подготовка аппаратной части}

Включение приборов рекомендуется производить в указанной последовательности.

\subsection{Подготовка Образца и термопары}

Подключите Образец, термопару и, возможно, эталонное сопротивление к разъёмам соединительного бокса согласно принципиальной схеме.

\subsection{Подготовка АС-4}

\begin{enumerate}

\item Включите прибор. Убедитесь, что он подключён к ПЭВМ.
\item Проверьте подключение измерительных проводов: оно должно соответствовать схеме Установки и режиму измерения.

\end{enumerate}


\subsection{Подготовка ТРМ-201}

\begin{enumerate}

\item Включите прибор. Убедитесь, что он подключён к ПЭВМ.
\item Проверьте подключение измерительных проводов: оно должно соответствовать схеме Установки и режиму измерения.

\end{enumerate}


\section{Подготовка программной части}

Запустите программу Установки.

Убедитесь в работоспособности Программы и всех аппаратных устройств. Для этого убедитесь, что Программа определяет и выводит на экран ПЭВМ температуру и сопротивления Образца. Откройте вкладку \CTL{Измерение} (она открыта сразу после запуска Программы). В текстовых полях (\CTL{Ток}, \CTL{Напряжение} и т.~д.), а также на графиках должны выводиться обработанные показания приборов. На графике производной температуры по времени $dT/dt$ показания могут выводиться с небольшой задержкой.

После того, как показания приборов начали отображаться на вкладке, убедитесь, что они лежат в ожидаемом диапазоне значений, что говорит о правильности подключения всех устройств и работы Установки. Если некоторые показания явно некорректные, проверьте качество соединений, положение кнопок <<Front/Rear>> мультиметров и пр.

\subsection{Параметры Образца}

Откройте закладку \CTL{Образец}. Данная вкладка содержит параметры измеряемого образца.

\subsubsection{Геометрические параметры}
\label{sec_geom_params}

Если возможно, введите некоторые геометрические параметры. Они будут использоваться для вычисления удельного сопротивления Образца. 

В поле \CTL{Расстояние между потенциальными контактами} введите соответствующее расстояние и его погрешность. В самом простом случае этого достаточно для вычисления удельного сопротивления.

Если образец имеет форму близкую к параллелипипеду, то введите соответствующие размеры в полях \CTL{Длина}, \CTL{Ширина} и \CTL{Толщина}, а также их абсолютные погрешности. При этом длина Образца не используется в расчётах, а ширина и толщина используются для расчёта поперечного сечения.

Допускается, хотя и не рекомендуется, не указывать погрешности геометрических параметров. В этом случае соответствующие поля ввода должны быть пустыми.

Если геометрические параметры Образца неизвестны, очистите все соответствующие поля. В этом случае удельное сопротивление не будет вычисляться автоматически, но на общую работу установки это не повлияет.


\subsubsection{Файлы}

В поле \CTL{Имя файла результатов} вводится путь и имя файла, в который будут записаны результаты измерений для последующего использования. Как правило, это поле обязательно для заполнения. Если поле пустое, то в процессе измерений их результаты не будут фиксироваться в файле, что как правило недопустимо.

Имя файла может включать в себя строчку \CMD{\%AUTODATE\%} (без кавычек). Тогда путь к реальному файлу будет в этом месте включать три
 подкаталога, соответствующих текущим году, месяцу и дню месяца. Например, если в поле введено \CMD{C:\textbackslash{}res\textbackslash{}\%AUTODATE\%\textbackslash{}res.txt}, и измерения проводятся 5 января 2012 года, то реальный путь к файлу будет \CMD{C:\textbackslash{}res\textbackslash{}2012\textbackslash{}01\textbackslash{}05\textbackslash{}res.txt}.

В поле \CTL{Имя файла трассировки} вводится путь и имя файла, в который будут записаны необработанные результаты всех произведённых измерений. Как правило, трассировка нужна только на этапе отладки Установки. Размер файла трассировки значительно превышает размер файла результатов, что может иногда привести к переполнению диска. Так же как и имя файла результатов, имя файла трассировки может включать строку \CMD{\%AUTODATE\%}.

В поле \CTL{Формат файлов} выбирается нужный формат всех создаваемых файлов. В настоящее время поддерживается два формата:

\begin{itemize}
\item \CTL{TXT} --- текстовый формат, в котором каждое измерение записывается в отдельной строке, внутри строки значения разделяются символом табуляции. Формат поддерживается большинством научных приложений.
\item \CTL{CSV} --- текстовый формат, в котором каждое измерение записывается в отдельной строке, внутри строки значения разделяются запятой. Формат поддерживается рядом офисных приложений, например Miscosoft Excel.
\end{itemize}

Флаг \CTL{Переписать файлы} указывает, нужно ли всякий раз в начале измерений переписывать файлы заново. Если флаг сброшен, все новые показания будут дописываться в конец файлов.

В поле \CTL{Комментарий} вводится произвольный фрагмент текста, который будет записан в начале файла результатов. Поле может быть пустым, но рекомендуется вводить в него краткое описание Образца и условия проведения измерений, это в будущем облегчит анализ результатов.


\subsection{Параметры измерения}

Откройте вкладку <<Параметры измерения>>. Содержимое данной вкладки определяет условия проведения измерений.

\subsubsection{Метод регистрации}
\label{sec_reg_method}

В данном разделе выбирается способ и частота регистрации измерений (см. описание принципа работы на стр.~\pageref{sec_registration_types}).

Если требуется регистрация с фиксированным временным интервалом, то выберите флаг \CTL{Временная зависимость}, после чего в поле \CTL{Временной шаг} введите значение интервала.

Если требуется регистрация с фиксированным температурным интервалом, то выберите флаг \CTL{Температурная зависимость}, после чего в поле \CTL{Температурный шаг} введите значение интервала.

Если требуется нерегулярная регистрация, то выберите флаг \CTL{Вручную}.

\subsubsection{Метод измерения сопротивления}

\label{sec_r_measures}

Сопротивление Образца измеряется 4-х контактным методом, для чего к Образцу подводятся два потенциальных и два токовых контакта. Сопротивление определяется одним из следующих способов:

\subsubsection{Вольтметром/ амперметром}

Используются вольтметр, амперметр и ИП. Амперметр и ИП включены последовательно с Образцом, амперметр подключён к токовым контактам Образца. Вольтметр включён параллельно с Образцом и подключён к его потенциальным контактам. Сопротивление Образца определяется как $R = V/I$, где $R$~--- искомое сопротивление, $V$~--- показания вольтметра, $I$~--- показания амперметра.

Данный способ наиболее универсальный и точный, но требует максимального количества задействованных мультиметров.

\subsubsection{Вольтметром/ вольтметром}

Используются 2 вольтметра, ИП и эталонное сопротивление номинала $R_2$, включённое последовательно с Образцом и ИП. Первый вольтметр подключён  к потенциальным контактам Образца, второй вольметр --- к выходам эталонного сопротивления. Сопротивление Образца определяется как $R = R_2 V_1/V_2$, где $R$~--- искомое сопротивление, $V_1$~--- показания первого вольтметра, $V_2$~--- показания второго вольтметра.

Данный способ требует максимального количества задействованных мультиметров, а также высококачественное эталонное сопротивление. В ряде случаев он может обеспечить повышенную относительную точность измерений, когда сравниваются результаты измерений одного и того же Образца.

\subsubsection{Вольтметром/ вручную}
\label{sec_voltmeter_manually}

Используются вольтметр и ИП, Образец включён последовательно с ИП, вольтметр подключён к потенциальным контактам Образца. Ток в цепи $I$ считается известным  и неизменным в течении всего эксперимента. Сопротивление Образца определяется как $R = V/I$, где $R$~--- искомое сопротивление, $V$~--- показания вольтметра.

Данный способ наименее точный. Если ИП сам не производит измерение тока в цепи, требуется ручное измерение тока.

\subsubsection{Омметром}

Используется омметр, подключённый к Образцу 4-х контактным способом. Сопротивление Образца определяется непосредственным считыванием показаний омметра.

Данный способ наиболее простой, но не годится в том случае, когда сопротивление Образца слишком мало или велико, то есть выходит за диапазон точных измерений омметра.

\bigskip 

Во всех вышеприведённых способах функции вольметра, амперметра или омметра выполняет мультметр, переведённый в соответствующий режим.


\chapter{Измерения}

После подготовки к работе аппаратной и программной частей Установки можно приступить к измерениям.

\section{Проведение измерений}

Выберите вкладку \CTL{Измерение}. Ещё раз убедитесь в том, что все приборы работают, величины температуры и сопротивления Образца измеряются и находятся в ожидаемом диапазоне значений.

После этого нажмите кнопку \CTL{Начать запись}. Установка начнёт регистрацию температуры и сопротивления.

Оператор управляет температурой образца, например увеличением напряжения на обмотке электропечи. Скорость изменения температуры определяется требованиями эксперимента.

Если выбран режим ручной регистрации (стр.~\pageref{sec_reg_type_manual}), то оператор должен самостоятльно нажимать кнопку \CTL{Снять точку}\label{sec_manual} всякий раз, когда требуется зафиксировать измерение.

Кнопка \CTL{Снять точку} доступна и в других режимах, при регистрации временной и температурной зависимостей. То есть даже при автоматической регистрации оператор может вручную зарегистрировать нужное измерение. Например, если выбранный временной интервал довольно велик, а оператор наблюдает <<интересное поведение>> Образца, то он может вручную зафиксировать текущее измерение, даже если временной интервал ещё не истёк.

\IMPORTANT{Если выбран режим ручного измерения тока (см. <<Вольтметром/ вручную>> на стр.~\pageref{sec_voltmeter_manually}), то оператор должен самостоятельно следить за стабильностью силы тока в цепи: его значение не должно отклоняться от введённого заранее значения. В противном случае измерения будут неправильными. Но если в качестве источника питания используется Agilent E3645A, то это ограничение отсутствует, поскольку данный ИП самостоятельно контролирует силу тока. Более того, в процессе измерения допускается изменять силу тока, выдаваемого E3645A, поскольку он также выступает в качестве амперметра и сигнализирует об изменении тока в цепи.} 

По окончании измерений нажмите кнопку \CTL{Остановить запись}, после чего в течении нескольких секунд программа зафиксирует все результаты в файлах. Файлы далее доступны для анализа.

Далее можно произвести новую серию измерений, или завершить работу Установки.

\section{Индикация измерений}

В процессе работы Установка производит снятие показаний приборов, обработку и отображение результатов на экране Программы. Частота, с которой производится опрос приборов, зависит от параметров измерения (см. раздел <<Метод регистрации>> на стр.~\pageref{sec_reg_method}), а также от скорости изменения температуры. Чем быстрее изменяется температура Образца, и чем меньше заданный температурный или временной интервал, чем чаще будут опрашиваться измерительные приборы.

Индикация измерений производится на вкладке \CTL{Измерение}.

\subsection{Диаграммы}

Четыре диаграммы показывают состояние основных регистрируемых величин:

\subsubsection{Зависимость $R(T)$}

На диаграмме изображается зависимость сопротивления Образца от температуры, измерения изображаются в виде зелёных точек, связанных друг с другом. Каждая залёная точка соответствует одному измерению, записанному в файл результатов. Если эксперимент достаточно продолжительный и точек становится слишком много, они автоматически прореживаются, разумеется только на графике.

Кроме того на этой же диаграмме отображается текущее значение сопротивление и температуры в виде небольшого числа сиреневых точек. Эти точки соответствуют промежуточным результатам, которые не подлежат регистрации в файле данных и нужны только для визуального контроля текущего состояния сборки.

\subsubsection{Зависимость R(t)}

На диаграмме изображается зависимость сопротивления Образца от времени, при этом по горизонтальной оси отложено количество измерений (а не время в секундах или иных непосредственных величинах).

Также на диаграмме изображается линейная аппроксимация зависимости $R(t)$ в виде фиолетовой линии. Эта линия даёт возможность оценить тренд изменения сопротивления.

\subsubsection{Зависимость T(t)}

На диаграмме изображается зависимость температуры Образца от времени, по горизонтальной оси отложено количество измерений. Также изображается линейная аппроксимация зависимости $T(t)$ в виде фиолетовой линии.

\subsubsection{Зависимость $\frac{dT}{dt}(t)$}

На диаграмме изображается зависимость скорости изменения температуры Образца от времени, по горизонтальной оси отложено количество измерений. Также изображается линейная аппроксимация зависимости $\frac{dT}{dt}(t)$ в виде фиолетовой линии.

Скорость изменения температуры вычисляется следующим образом. Делается несколько измерений температуры, по ним вычисляется линейная аппроксимация. Скорость изменения температуры определяется как угол наклона этой аппроксимации.

На данном графике значения могут появляться с небольшой задержкой, вызванной тем, что скорость изменения температуры вычисляется спустя некоторое минимальное число измерений.

\subsection{Текстовые поля}

Также индикация измерений производится в текстовых полях, вместе озаглавленных как \CTL{Результаты измерения}. Здесь выводятся следующие текущие значения:

\begin{itemize}
\item ток через Образец;
\item падение напряжения на потенциальных контактах Образца;
\item сопротивление между потенциальными контактами;
\item выделаемая тепловая мощность на Образце между потенциальными контактами (не на всём Образце!);
\item температура Образца и скорость её изменения.
\end{itemize}

Все величины сопровождаются инструментальной погрешностью.

\section{Файл результатов}

В файле результатов в начале идёт строка с комментарием (стр.~\pageref{sec_dut_comment}), далее идут следующие поля:

\begin{enumerate}
\item \CMD{Date/Time} --- местные дата и время, в которое было произведено измерение, в формате \mbox{\CMD{ГГГГ-ДД-ММ чч:мм:сс}}.
\item \CMD{T} --- температура Образца в К.
\item \CMD{+/-} --- погрешность определения температуры в К. Здесь и далее под погрешностью подразумевается абсолютная инструментальная погрешность.
\item \CMD{dT/dt} --- скорость изменения температуры в К/мин.
\item \CMD{I} --- ток через Образец в мА.
\item \CMD{+/-} --- погрешность определения тока в мА.
\item \CMD{U} --- падение напряжения на потенциальных контактах Образца в мВ.
\item \CMD{+/-} --- погрешность определения напряжения в мВ.
\item \CMD{R} --- сопротивление между потенциальными контактами Образца в Ом.
\item \CMD{+/-} --- погрешность определения сопротивления в Ом.
\item \CMD{Rho} --- удельное сопротивление между потенциальными контактами Образца в Ом${}\cdot{}$см. Если геометрические параметры Образца не были указаны (стр.~\pageref{sec_geom_params}), данное поле будет пустым.
\item \CMD{+/-} --- погрешность определения удельного сопротивления в Ом${}\cdot{}$см. Так же как и предыдущее, данное поле будет пустым при невозможности определения удельного сопротивления.
\item \CMD{Manual} --- если данная точка была снята вручную (стр.~\pageref{sec_manual}), в данном поле будет значение \CMD{true}, в противном случае поле будет пустым.
\end{enumerate}

\chapter{Завершение работы}

Для завершения работы Установки нажмите кнопку \CTL{Выход} в панели Программы. Программа произведёт сброс всех устройств в исходное состояние и закончит работу. Далее можно приступать к выключению аппаратной части.

\section{Отключение приборов}

Отключение приборов рекомендуется производить в указанной последовательности.

\subsection{Отключение мультиметров 34410A/34401A}

\begin{enumerate}

\item Установите значения тока и напряжения в минимальные положения.
\item Выключите источник питания.

\end{enumerate}


\subsection{Отключение источника питания}

\begin{enumerate}

\item Установите значения тока и напряжения в минимальные положения.
\item Выключите источник питания.

\end{enumerate}


\subsection{Отключение соединительного бокса}

Выключите питание соединительного бокса.

\bigskip

После этого можно отсоединить приборы и сборку с Образцом.

\chapter{Настройка Программы}

При первом запуске Программы, а также при всяком изменении аппаратной части, требуется произвести настройку или перенастройку программной части, чтобы обеспечить связь с аппаратной частью и, возможно, её калибровку.

\section{Файл конфигурации}

Все свои настройки программа сохраняет в обычном файле с именем \FILENAME{\PROGNAME{}.ini}, который размещается в текущем каталоге. Файл имеет текстовый формат, и при необходимости его можно редактировать в произвольном текстовом редакторе. Если файл конфигурации отсутствует при запуске Программы, он будет автоматически создан.

Обратите внимание: файл конфигурации всегда располагается в {\it текущем} каталоге, а не в каталоге, где расположена сама программа. Эти каталоги могут совпадать, но вообще говоря они могут быть разными. Это позволяет запускать одну и ту же программу, но с разными настройками для разных экспериментов.

Рассмотрим пример. Пусть имеется две измерительные установки, различающиеся, например, сборкой, в которой установлен Образец. Таким образом имеем одинаковый набор мультиметров и источников питания, но в разных экспериментах у нас разные термопары, имеющие разную калибровку. Чтобы каждый раз при переключении не менять настройки, мы делаем следующее:

\begin{enumerate}
\item Создаём два разных каталога, например \FILENAME{C:\textbackslash{}config\textbackslash{}exp1} и \FILENAME{C:\textbackslash{}config\textbackslash{}exp2}.

\item Сама программа путь будет расположена в каталоге \FILENAME{C:\textbackslash{}prog\textbackslash{}assembly007}.

\item Для работы с первой сборкой и переходим в каталог \FILENAME{C:\textbackslash{}config\textbackslash{}exp1} и запускаем программу оттуда. В этом же каталоге будет создан файл конфигурации \FILENAME{\PROGNAME{}.ini}.

\item Для работы со второй сборкой и переходим в каталог \FILENAME{C:\textbackslash{}config\textbackslash{}exp2} и производим аналогичные действия.

\end{enumerate}

Если мы работаем в операционной системе семейства Windows, то удобно будет создать ярлыки для каждой из сборок. В свойствах ярлыка укажите разные рабочие каталоги, тогда при выборе ярлыка будет устанавливаться соответствующий текущий каталог, в котором программа будет искать конфигурационный файл.


\section{Настройка измерения сопротивления}

Откройте вкладку \CTL{Параметры измерения сопротивления} Программы. Здесь оператор вводит параметры (адреса и пр.) устройств, которые используются для определения сопротивления Образца.

\subsection{Блок реле}

В данном разделе --- параметры блока реле МВУ-8, содержащего переключатели S1--S4.

В поле \CTL{Тип термопары} выбирается один из поддерживаемых типов. Расшифровку типов термопар см. в таблице~\ref{tab_tc_types}

\begin{table*}
\begin{center}
\caption{Типы термопар}
\begin{tabular}{cl}
\hline \hline
B & Платина 30\% + Родий / Платина 60\% + Родий \\
C & Вольфрам 5\% + Рений / Вольфрам 26\% + Рений \\
E & Хромель / Константан \\
J & Железо / Константан \\
K & Хромель / Алюмель \\
N & Нихросил / Нисил \\
R & Платина 13\% + Родий / Платина \\
S & Платина 10\% + Родий / Платина \\
T & Медь / Константан \\
\hline \hline
\end{tabular}
\label{tab_tc_types}
\end{center}
\end{table*}

В поле \CTL{Опорная температура} вводится температура опорного спая термопары, которая считается постоянной на всём протяжении эксперимента. Например, если опорный спай погружён в жидкий азот, поле должно иметь значение $77.4$.

Флаг \CTL{Инв. полярность} меняет полярность подключения термопары. Если температура, отображаемая Программой, некорректная (слишком большая или слишком маленькая), это может быть вызвано тем, что термопара подключена с неправильной полярностью. Менять физическое подключение в этом случае не нужно, достаточно поменять значение флага на противоположное.

В поле \CTL{Выражение для коррекции} вводится арифметическое выражение, которое меняет некоторым образом температуру, полученную непосредственно от термопары. Как известно, всякая термопара требует калибровки, по результатам которой показания должны быть скорректированы для устранения систематической погрешности. В данном поле и вводится это выражение для коррекции.

Выражение включает в себя переменную, которая содержит в себе температуру, полученную непосредственно от термопары, одно или несколько чисел и арифметические операции, выражаемые символами \CMD{+}, \CMD{-}, \CMD{*} и \CMD{/}. Порядок операций~--- как принято в арифметике, для изменения порядка используются круглые скобки. Переменная обозначается произвольной латинской буквой. Между операндами допускается произвольное количество пробелов для улучшения читабельности.

{\bf Пример. } Пусть показания термопары дают температуру, увеличенную на $0.5$~К по сравнению с реальной, корректировка должна уменьшать исходное значение на $0.5$. Тогда выражение для корректировки будет \mbox{\CMD{x - 0.5}} (выражение вводится в поле без кавычек).

\bigskip

Для облегчения калибровки Программа может автоматически составить выражение для коррекции по двум контрольным измерениям температуры:

\begin{enumerate}
\item Сбросьте текущую калибровку (очистите поле \CTL{Выражение для коррекции}) и перезапустите Программу.
\item Выполните два контрольных измерения температуры, для этого погрузите рабочий спай термопары в среды с известной температурой (например, жидкий азот и тающий лёд). Значения температуры возьмите из поля \CTL{Температура} на вкладке \CTL{Измерение}.
\item Перейдите вновь на вкладку \CTL{Параметры установки} и нажмите кнопку \CTL{Калибровка}.
\item Введите значения контрольных температур и соответствующих им показаниям термопары, после чего нажмите кнопку \CTL{Расчёт}.
\end{enumerate}

Программа вычислит подходящее выражение для коррекции и запишет его в поле. После этого оператор может в случае необходимости вручную изменить это выражение.

\subsection{Вольтметр/омметр на образце}
\label{sec_mm1_config}

В данном разделе --- параметры мультиметра MM1 (см. раздел <<Принципиальная схема>> на стр.~\pageref{sec_schematic_diagram}).

В поле \CTL{Тип термопары} выбирается один из поддерживаемых типов. Расшифровку типов термопар см. в таблице~\ref{tab_tc_types}

\begin{table*}
\begin{center}
\caption{Типы термопар}
\begin{tabular}{cl}
\hline \hline
B & Платина 30\% + Родий / Платина 60\% + Родий \\
C & Вольфрам 5\% + Рений / Вольфрам 26\% + Рений \\
E & Хромель / Константан \\
J & Железо / Константан \\
K & Хромель / Алюмель \\
N & Нихросил / Нисил \\
R & Платина 13\% + Родий / Платина \\
S & Платина 10\% + Родий / Платина \\
T & Медь / Константан \\
\hline \hline
\end{tabular}
\label{tab_tc_types}
\end{center}
\end{table*}

В поле \CTL{Опорная температура} вводится температура опорного спая термопары, которая считается постоянной на всём протяжении эксперимента. Например, если опорный спай погружён в жидкий азот, поле должно иметь значение $77.4$.

Флаг \CTL{Инв. полярность} меняет полярность подключения термопары. Если температура, отображаемая Программой, некорректная (слишком большая или слишком маленькая), это может быть вызвано тем, что термопара подключена с неправильной полярностью. Менять физическое подключение в этом случае не нужно, достаточно поменять значение флага на противоположное.

В поле \CTL{Выражение для коррекции} вводится арифметическое выражение, которое меняет некоторым образом температуру, полученную непосредственно от термопары. Как известно, всякая термопара требует калибровки, по результатам которой показания должны быть скорректированы для устранения систематической погрешности. В данном поле и вводится это выражение для коррекции.

Выражение включает в себя переменную, которая содержит в себе температуру, полученную непосредственно от термопары, одно или несколько чисел и арифметические операции, выражаемые символами \CMD{+}, \CMD{-}, \CMD{*} и \CMD{/}. Порядок операций~--- как принято в арифметике, для изменения порядка используются круглые скобки. Переменная обозначается произвольной латинской буквой. Между операндами допускается произвольное количество пробелов для улучшения читабельности.

{\bf Пример. } Пусть показания термопары дают температуру, увеличенную на $0.5$~К по сравнению с реальной, корректировка должна уменьшать исходное значение на $0.5$. Тогда выражение для корректировки будет \mbox{\CMD{x - 0.5}} (выражение вводится в поле без кавычек).

\bigskip

Для облегчения калибровки Программа может автоматически составить выражение для коррекции по двум контрольным измерениям температуры:

\begin{enumerate}
\item Сбросьте текущую калибровку (очистите поле \CTL{Выражение для коррекции}) и перезапустите Программу.
\item Выполните два контрольных измерения температуры, для этого погрузите рабочий спай термопары в среды с известной температурой (например, жидкий азот и тающий лёд). Значения температуры возьмите из поля \CTL{Температура} на вкладке \CTL{Измерение}.
\item Перейдите вновь на вкладку \CTL{Параметры установки} и нажмите кнопку \CTL{Калибровка}.
\item Введите значения контрольных температур и соответствующих им показаниям термопары, после чего нажмите кнопку \CTL{Расчёт}.
\end{enumerate}

Программа вычислит подходящее выражение для коррекции и запишет его в поле. После этого оператор может в случае необходимости вручную изменить это выражение.

\subsection{Амперметр/вольтметр на эталоне}

В данном разделе --- параметры мультиметра MM2 (см. раздел <<Принципиальная схема>> на стр.~\pageref{sec_schematic_diagram}). Способ настройки~--- такой же, как для мультиметра MM1.

\IMPORTANT{Если мультиметр MM2 не используется в схеме Установки (например, при ручном измерении тока), то поле адреса должно быть пустым.}

\subsection{Источник питания}

В данном разделе --- параметры ИП I1 (см. раздел <<Принципиальная схема>> на стр.~\pageref{sec_schematic_diagram}), если в его качестве используется Agilent E3645A или аналогичный.

\IMPORTANT{Если используется ИП другого типа, все поля в данном разделе должны быть пустыми.}

В поле \CTL{Тип термопары} выбирается один из поддерживаемых типов. Расшифровку типов термопар см. в таблице~\ref{tab_tc_types}

\begin{table*}
\begin{center}
\caption{Типы термопар}
\begin{tabular}{cl}
\hline \hline
B & Платина 30\% + Родий / Платина 60\% + Родий \\
C & Вольфрам 5\% + Рений / Вольфрам 26\% + Рений \\
E & Хромель / Константан \\
J & Железо / Константан \\
K & Хромель / Алюмель \\
N & Нихросил / Нисил \\
R & Платина 13\% + Родий / Платина \\
S & Платина 10\% + Родий / Платина \\
T & Медь / Константан \\
\hline \hline
\end{tabular}
\label{tab_tc_types}
\end{center}
\end{table*}

В поле \CTL{Опорная температура} вводится температура опорного спая термопары, которая считается постоянной на всём протяжении эксперимента. Например, если опорный спай погружён в жидкий азот, поле должно иметь значение $77.4$.

Флаг \CTL{Инв. полярность} меняет полярность подключения термопары. Если температура, отображаемая Программой, некорректная (слишком большая или слишком маленькая), это может быть вызвано тем, что термопара подключена с неправильной полярностью. Менять физическое подключение в этом случае не нужно, достаточно поменять значение флага на противоположное.

В поле \CTL{Выражение для коррекции} вводится арифметическое выражение, которое меняет некоторым образом температуру, полученную непосредственно от термопары. Как известно, всякая термопара требует калибровки, по результатам которой показания должны быть скорректированы для устранения систематической погрешности. В данном поле и вводится это выражение для коррекции.

Выражение включает в себя переменную, которая содержит в себе температуру, полученную непосредственно от термопары, одно или несколько чисел и арифметические операции, выражаемые символами \CMD{+}, \CMD{-}, \CMD{*} и \CMD{/}. Порядок операций~--- как принято в арифметике, для изменения порядка используются круглые скобки. Переменная обозначается произвольной латинской буквой. Между операндами допускается произвольное количество пробелов для улучшения читабельности.

{\bf Пример. } Пусть показания термопары дают температуру, увеличенную на $0.5$~К по сравнению с реальной, корректировка должна уменьшать исходное значение на $0.5$. Тогда выражение для корректировки будет \mbox{\CMD{x - 0.5}} (выражение вводится в поле без кавычек).

\bigskip

Для облегчения калибровки Программа может автоматически составить выражение для коррекции по двум контрольным измерениям температуры:

\begin{enumerate}
\item Сбросьте текущую калибровку (очистите поле \CTL{Выражение для коррекции}) и перезапустите Программу.
\item Выполните два контрольных измерения температуры, для этого погрузите рабочий спай термопары в среды с известной температурой (например, жидкий азот и тающий лёд). Значения температуры возьмите из поля \CTL{Температура} на вкладке \CTL{Измерение}.
\item Перейдите вновь на вкладку \CTL{Параметры установки} и нажмите кнопку \CTL{Калибровка}.
\item Введите значения контрольных температур и соответствующих им показаниям термопары, после чего нажмите кнопку \CTL{Расчёт}.
\end{enumerate}

Программа вычислит подходящее выражение для коррекции и запишет его в поле. После этого оператор может в случае необходимости вручную изменить это выражение.

\section{Настройка измерения температуры}

Откройте вкладку \CTL{Параметры измерения температуры} Программы. Здесь оператор вводит параметры (адреса и пр.) устройств, которые используются для определения температуры Образца.

\subsection{Способ подключения термопары}

В данном разделе оператор выбирает один из способов подключения термопары (см. раздел <<Определение температуры>> на стр.~\pageref{sec_t_measures}).

\subsection{Вольтметр на термопаре}

В данном разделе --- параметры мультиметра MM3 (см. раздел <<Принципиальная схема>> на стр.~\pageref{sec_schematic_diagram}). Способ настройки~--- такой же, как для мультиметра MM1 (стр.~\pageref{sec_mm1_config}). Если для определения температуры используется ТРМ-201, то данный раздел недоступен для редактирования).

\subsection{Измеритель-регулятор ТРМ-201}

В данном разделе --- параметры ТРМ-201, используемого для определения температуры. Если для определения температуры используется вольтметр, то данный раздел недоступен для редактирования).

В поле \CTL{Тип термопары} выбирается один из поддерживаемых типов. Расшифровку типов термопар см. в таблице~\ref{tab_tc_types}

\begin{table*}
\begin{center}
\caption{Типы термопар}
\begin{tabular}{cl}
\hline \hline
B & Платина 30\% + Родий / Платина 60\% + Родий \\
C & Вольфрам 5\% + Рений / Вольфрам 26\% + Рений \\
E & Хромель / Константан \\
J & Железо / Константан \\
K & Хромель / Алюмель \\
N & Нихросил / Нисил \\
R & Платина 13\% + Родий / Платина \\
S & Платина 10\% + Родий / Платина \\
T & Медь / Константан \\
\hline \hline
\end{tabular}
\label{tab_tc_types}
\end{center}
\end{table*}

В поле \CTL{Опорная температура} вводится температура опорного спая термопары, которая считается постоянной на всём протяжении эксперимента. Например, если опорный спай погружён в жидкий азот, поле должно иметь значение $77.4$.

Флаг \CTL{Инв. полярность} меняет полярность подключения термопары. Если температура, отображаемая Программой, некорректная (слишком большая или слишком маленькая), это может быть вызвано тем, что термопара подключена с неправильной полярностью. Менять физическое подключение в этом случае не нужно, достаточно поменять значение флага на противоположное.

В поле \CTL{Выражение для коррекции} вводится арифметическое выражение, которое меняет некоторым образом температуру, полученную непосредственно от термопары. Как известно, всякая термопара требует калибровки, по результатам которой показания должны быть скорректированы для устранения систематической погрешности. В данном поле и вводится это выражение для коррекции.

Выражение включает в себя переменную, которая содержит в себе температуру, полученную непосредственно от термопары, одно или несколько чисел и арифметические операции, выражаемые символами \CMD{+}, \CMD{-}, \CMD{*} и \CMD{/}. Порядок операций~--- как принято в арифметике, для изменения порядка используются круглые скобки. Переменная обозначается произвольной латинской буквой. Между операндами допускается произвольное количество пробелов для улучшения читабельности.

{\bf Пример. } Пусть показания термопары дают температуру, увеличенную на $0.5$~К по сравнению с реальной, корректировка должна уменьшать исходное значение на $0.5$. Тогда выражение для корректировки будет \mbox{\CMD{x - 0.5}} (выражение вводится в поле без кавычек).

\bigskip

Для облегчения калибровки Программа может автоматически составить выражение для коррекции по двум контрольным измерениям температуры:

\begin{enumerate}
\item Сбросьте текущую калибровку (очистите поле \CTL{Выражение для коррекции}) и перезапустите Программу.
\item Выполните два контрольных измерения температуры, для этого погрузите рабочий спай термопары в среды с известной температурой (например, жидкий азот и тающий лёд). Значения температуры возьмите из поля \CTL{Температура} на вкладке \CTL{Измерение}.
\item Перейдите вновь на вкладку \CTL{Параметры установки} и нажмите кнопку \CTL{Калибровка}.
\item Введите значения контрольных температур и соответствующих им показаниям термопары, после чего нажмите кнопку \CTL{Расчёт}.
\end{enumerate}

Программа вычислит подходящее выражение для коррекции и запишет его в поле. После этого оператор может в случае необходимости вручную изменить это выражение.

\subsection{Термопара}

В данном разделе --- параметры термопары, используемой для определения температуры Образца.

В поле \CTL{Тип термопары} выбирается один из поддерживаемых типов. Расшифровку типов термопар см. в таблице~\ref{tab_tc_types}

\begin{table*}
\begin{center}
\caption{Типы термопар}
\begin{tabular}{cl}
\hline \hline
B & Платина 30\% + Родий / Платина 60\% + Родий \\
C & Вольфрам 5\% + Рений / Вольфрам 26\% + Рений \\
E & Хромель / Константан \\
J & Железо / Константан \\
K & Хромель / Алюмель \\
N & Нихросил / Нисил \\
R & Платина 13\% + Родий / Платина \\
S & Платина 10\% + Родий / Платина \\
T & Медь / Константан \\
\hline \hline
\end{tabular}
\label{tab_tc_types}
\end{center}
\end{table*}

В поле \CTL{Опорная температура} вводится температура опорного спая термопары, которая считается постоянной на всём протяжении эксперимента. Например, если опорный спай погружён в жидкий азот, поле должно иметь значение $77.4$.

Флаг \CTL{Инв. полярность} меняет полярность подключения термопары. Если температура, отображаемая Программой, некорректная (слишком большая или слишком маленькая), это может быть вызвано тем, что термопара подключена с неправильной полярностью. Менять физическое подключение в этом случае не нужно, достаточно поменять значение флага на противоположное.

В поле \CTL{Выражение для коррекции} вводится арифметическое выражение, которое меняет некоторым образом температуру, полученную непосредственно от термопары. Как известно, всякая термопара требует калибровки, по результатам которой показания должны быть скорректированы для устранения систематической погрешности. В данном поле и вводится это выражение для коррекции.

Выражение включает в себя переменную, которая содержит в себе температуру, полученную непосредственно от термопары, одно или несколько чисел и арифметические операции, выражаемые символами \CMD{+}, \CMD{-}, \CMD{*} и \CMD{/}. Порядок операций~--- как принято в арифметике, для изменения порядка используются круглые скобки. Переменная обозначается произвольной латинской буквой. Между операндами допускается произвольное количество пробелов для улучшения читабельности.

{\bf Пример. } Пусть показания термопары дают температуру, увеличенную на $0.5$~К по сравнению с реальной, корректировка должна уменьшать исходное значение на $0.5$. Тогда выражение для корректировки будет \mbox{\CMD{x - 0.5}} (выражение вводится в поле без кавычек).

\bigskip

Для облегчения калибровки Программа может автоматически составить выражение для коррекции по двум контрольным измерениям температуры:

\begin{enumerate}
\item Сбросьте текущую калибровку (очистите поле \CTL{Выражение для коррекции}) и перезапустите Программу.
\item Выполните два контрольных измерения температуры, для этого погрузите рабочий спай термопары в среды с известной температурой (например, жидкий азот и тающий лёд). Значения температуры возьмите из поля \CTL{Температура} на вкладке \CTL{Измерение}.
\item Перейдите вновь на вкладку \CTL{Параметры установки} и нажмите кнопку \CTL{Калибровка}.
\item Введите значения контрольных температур и соответствующих им показаниям термопары, после чего нажмите кнопку \CTL{Расчёт}.
\end{enumerate}

Программа вычислит подходящее выражение для коррекции и запишет его в поле. После этого оператор может в случае необходимости вручную изменить это выражение.

\section{Завершение настройки}

По окончании настройки Программа должна быть перезапущена, чтобы именения вошли в силу.

\chapter{Устранение неиправностей}

\section{Общие рекомендации}

Если при запуске Программы результаты измерений не отображаются, то рекомендуется выполнить следующие действия:

\begin{enumerate}
\item Убедитесь в правильности всех подключений.
\item Убедитесь, что выбранный в Программе метод измерения (стр.~\pageref{sec_r_measure_config}) соответствует схеме Установки и подключениям.
\item Убедитесь, что Программа установила связь со всеми устройствами. Для этого зайдите на вкладку \CTL{Параметры установки} и выполните опрос всех используемых устройств по очереди.
\end{enumerate}

В случае обнаружения неверных настроек (например, выбран неверный метод измерения или введён неправильный адрес устройства), выполните их коррекцию и перезапустите Программу.

Если все вышеуказанные действия не помогли, необходимо ознакомиться с содержимым файла протокола для детального выяснения неисправности.

\section{Файл протокола}

Программа записывает информацию о всех обнаруженных неисправностях аппаратной части в файл протокола с именем \FILENAME{\PROGNAME{}.log}, который размещается в текущем каталоге. Файл всегда пополняется, то есть при обнаружении ошибки новые записи добавляются в конец файла.

Файл имеет простой текстовый формат, в котором каждая запись имеет следующий вид:

\CMD{Время Важность Модуль Описание}

Здесь <Время>~--- точные дата и время обнаружения неисправности. <Важность>~--- важность ситуации, которая может принимать следующие значения:

\begin{itemize}
\item \CTL{critical} --- критическая ошибка;
\item \CTL{error} --- важная ошибка;
\item \CTL{warning} -- предупреждение о возможной ошибке;
\item \CTL{info} --- информационное сообщение, не сигнализирующее об ошибке;
\item \CTL{debug} --- отладочное сообщение, предназначенное для отладки программы.
\end{itemize}

<Модуль>~--- имя модуля Программы, в котором обнаружена ошибка. <Описание>~--- произвольное текстовое описание, детально раскрывающее суть и местоположение ошибки. Описание может распологаться на нескольких строках файла протокола.

Если файл протокола отсутствует при работающей программе, значит в процессе работы ещё не возникало ни одной ошибки.

Файл протокола можно безопасно удалять, он будет вновь создан при первой же возникшей ошибке.

Поскольку файл протокола постоянно пополняется, оператору следует время от времени удалять его во избежание переполнения диска.


\end{document}
