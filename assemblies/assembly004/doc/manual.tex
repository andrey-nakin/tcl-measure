\documentclass[12pt, a4paper, twocolumn]{book}
\usepackage[utf8]{inputenc}
\usepackage[russian]{babel}
\usepackage{hyperref}

%\makeindex

%\newcommand{\COMMANDREF}[1]{{\tt \hyperref[#1]{#1}}}
%\newcommand{\VISACOMMANDREF}[1]{{\tt \mbox{#1}}\index{#1}}
%\newcommand{\TCLCOMMANDREF}[1]{{\tt \mbox{#1}}\index{#1}}

\title{Установка №~4 для измерения удельного сопротивления при постоянной температуре. Руководство пользователя}
\author{Накин~А.~В.}

\begin{document}

\maketitle

\tableofcontents

\chapter{Общие сведения}

Установка №~4 (далее~--- Установка) представляет собой программно-аппаратный комплекс для измерения удельного сопротивления низкоомных материалов (далее~--- Образцов) при постоянной температуре.

Установка производит съём показаний измерительных приборов, вычисление значения сопротивления со всеми сопутствующими погрешностями и запись результатов в файл. В зависимости от конфигурации Установка может регулировать ток питания в указанном диапазоне, таким образом измеряя зависимость сопротивления Образца от тока.

Установка не управляет температурой Образца и не измеряет её.

\section{Состав аппаратной части}

Аппаратная часть Установки состоит из следующих приборов.

\begin{itemize}

\item Мультиметр 34410A компании Agilent (2~шт.). Предназначен для измерения падения напряжения на Образце и силы тока в цепи.

\item Управляемый источник питания постоянного тока (далее~--- ИП) E3645A компании Agilent (0-1~шт.). Предназначен для подачи тока в цепь с Образцом. Управляется ПЭВМ и позволяет автоматизированное измерение сопротивления в диапазоне токов. В зависимости от конфигурации Установки может отсутствовать.

\item Произвольный неуправляемый ИП постоянного тока (0-1~шт.). Предназначен для подачи тока в цепь с Образцом. Используется вместо E3645A в тех случах, когда требуется повышенная стабильность тока питания и не требуется измерение сопротивления в диапазоне токов.

\item Блок из 8-ми управляемых 2-х позиционных реле МВУ-8 компании ОВЕН (1~шт.). Предназначен для коммутации приборов и Образца. Управляется ПЭВМ посредством прибора АС-4.

\item Преобразователь интерфейсов USB/RS-485 АС-4 компании ОВЕН (1~шт.). Предназначен для подключения МВУ-8 к ПЭВМ через USB интерфейс.

\item Эталонное сопротивление (0-1~шт.). Предназначено для измерения тока в цепи посредством измерения падения напряжения на сопротивлении с известным номиналом. В зависимости от конфигурации Установки может отсутствовать, тогда ток в цепи измеряется мультиметром 34410A, работающим в режиме амперметра.

\end{itemize}

\section{Состав программной части}

Программная часть Установки состоит из следующих компонентов.

\begin{itemize}

\item Программа Установки, состоящая из нескольких модулей, написанных на языке Tcl.

\item Интерпретатор языка Tcl и стандартные библиотеки: tcllib, Thread, Tk, Ttk.

\item Библиотека tcl-measure с поддержкой устройств, многопоточности и пр.

\item Библиотека tclvisa для доступа к библиотеке VISA из Tcl.

\item Библиотека, предоставляющая программный интерфейс VISA для доступа к приборам 34410A и E3645A.

\item Драйвер устройства АС-4, представляющий устройство в виде COM-порта.

Сведения об установке интерпретатора Tcl, библиотек для него, библиотеки VISA изложены в соответствующих документах. Сведения об установке драйвера АС-4 см. в документации к прибору или на сайте производителя\footnote{\href{http://www.owen.ru/}{http://www.owen.ru/}}.

\end{itemize}

\chapter{Подготовка к работе}

\section{Подготовка аппаратной части}

\subsection{Подготовка мультиметров Agilent 34410A}

Мультметры должны быть включены и подключены к ПЭВМ посредством USB интерфейса.

Выходы первого мультиметра должны быть установлены в положение для измерения напряжения

Если Установка сконфигурирована для измерения тока в цепи посредством непосредственного измерения, выходы второго мультиметра должны быть установлены в положение для измерения тока. В противном случае выходы должны быть установлены в положение для измерения напряжения.

\subsection{Подготовка ИП Agilent E3645A}

ИП должен быть включен и подключён к ПЭВМ посредством интерфейса RS-232. Выходы ИП должны быть подключены к соответствующим разъёмам Установки.

ИП должен быть настроен для работы по интерфейсу RS-232. Согласно настройкам по умолчанию (<<заводским>>) он работает по интерфейсу GPIB. В этом случае прибор требует перенастройки на использование RS-232. Изменённые настройки сохраняются в энергонезависимой памяти устройства и не теряются при выключении питания. См. документацию к Agilent E3645A для детальных инструкций.

\end{document}
