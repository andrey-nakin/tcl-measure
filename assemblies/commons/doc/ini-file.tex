Все свои настройки программа сохраняет в обычном файле с именем \FILENAME{\PROGNAME{}.ini}, который размещается в текущем каталоге. Файл имеет текстовый формат, и при необходимости его можно редактировать в произвольном текстовом редакторе. Если файл конфигурации отсутствует при запуске Программы, он будет автоматически создан.

Обратите внимание: файл конфигурации всегда располагается в {\it текущем} каталоге, а не в каталоге, где расположена сама программа. Эти каталоги могут совпадать, но вообще говоря они могут быть разными. Это позволяет запускать одну и ту же программу, но с разными настройками для разных экспериментов.

Рассмотрим пример. Пусть имеется две измерительные установки, различающиеся, например, сборкой, в которой установлен Образец. Таким образом имеем одинаковый набор мультиметров и источников питания, но в разных экспериментах у нас разные термопары, имеющие разную калибровку. Чтобы каждый раз при переключении не менять настройки, мы делаем следующее:

\begin{enumerate}
\item Создаём два разных каталога, например \FILENAME{C:\textbackslash{}config\textbackslash{}exp1} и \FILENAME{C:\textbackslash{}config\textbackslash{}exp2}.

\item Сама программа путь будет расположена в каталоге \FILENAME{C:\textbackslash{}prog\textbackslash{}assembly007}.

\item Для работы с первой сборкой и переходим в каталог \FILENAME{C:\textbackslash{}config\textbackslash{}exp1} и запускаем программу оттуда. В этом же каталоге будет создан файл конфигурации \FILENAME{\PROGNAME{}.ini}.

\item Для работы со второй сборкой и переходим в каталог \FILENAME{C:\textbackslash{}config\textbackslash{}exp2} и производим аналогичные действия.

\end{enumerate}

Если мы работаем в операционной системе семейства Windows, то удобно будет создать ярлыки для каждой из сборок. В свойствах ярлыка укажите разные рабочие каталоги, тогда при выборе ярлыка будет устанавливаться соответствующий текущий каталог, в котором программа будет искать конфигурационный файл.
