\label{sec_t_measures}

Температура Образца измеряется посредством термопары. Имеется два способа подключения термопары:

\subsubsection*{К вольтметру}

Термопара имеет два спая: рабочий и опорный. Рабочий спай помещается рядом с Образцом, а опорный спай погружается в термостабильную среду с известной температурой (например, жидкий азот). Важно, чтобы температура этой среды не менялась в течении всего эксперимента.

Свободные концы подключаются непосредственно к вольтметру, по показаниям которого определяется разность температур рабочего и опорного спаев.

Данный способ позволяет производить температурные измерения с достаточной точностью (до 0,1~К) и высокой скоростью (до 25~отсчётов в секунду) без потери точности.

\subsubsection*{К ТРМ-201}

Термопара имеет один единственный рабочий спай и подключается к прибору ТРМ-201. Прибор сам определяет температуру <<холодного>> спая (свободных концов термопары), измеряет разность потенциалов на нём и преобразует её в абсолютное температурное значение.

Данный способ возволяет высвободить один мультиметр и не требует наличия термостабильной среды. Недостатки: большая погрешность определения температуры (до 0,5\%) и невысокая скорость измерения (не чаще одного отсчёта в секунду).