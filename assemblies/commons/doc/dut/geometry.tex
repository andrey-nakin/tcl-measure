Если возможно, введите некоторые геометрические параметры. Они будут использоваться для вычисления удельного сопротивления Образца. 

В поле \CTL{Расстояние между потенциальными контактами} введите соответствующее расстояние и его погрешность. В самом простом случае этого достаточно для вычисления удельного сопротивления.

Если образец имеет форму близкую к параллелипипеду, то введите соответствующие размеры в полях \CTL{Длина}, \CTL{Ширина} и \CTL{Толщина}, а также их абсолютные погрешности. При этом длина Образца не используется в расчётах, а ширина и толщина используются для расчёта поперечного сечения.

Допускается, хотя и не рекомендуется, не указывать погрешности геометрических параметров. В этом случае соответствующие поля ввода должны быть пустыми.

Если геометрические параметры Образца неизвестны, очистите все соответствующие поля. В этом случае удельное сопротивление не будет вычисляться автоматически, но на общую работу установки это не повлияет.
