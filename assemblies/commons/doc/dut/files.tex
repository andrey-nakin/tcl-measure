В поле \CTL{Имя файла результатов} вводится путь и имя файла, в который будут записаны результаты измерений для последующего использования. Как правило, это поле обязательно для заполнения. Если поле пустое, то в процессе измерений их результаты не будут фиксироваться в файле, что как правило недопустимо.

Имя файла может включать в себя строчку \CMD{\%AUTODATE\%} (без кавычек). Тогда путь к реальному файлу будет в этом месте включать три
 подкаталога, соответствующих текущим году, месяцу и дню месяца. Например, если в поле введено \CMD{C:\textbackslash{}res\textbackslash{}\%AUTODATE\%\textbackslash{}res.txt}, и измерения проводятся 5 января 2012 года, то реальный путь к файлу будет \CMD{C:\textbackslash{}res\textbackslash{}2012\textbackslash{}01\textbackslash{}05\textbackslash{}res.txt}.

В поле \CTL{Имя файла трассировки} вводится путь и имя файла, в который будут записаны необработанные результаты всех произведённых измерений. Как правило, трассировка нужна только на этапе отладки Установки. Размер файла трассировки значительно превышает размер файла результатов, что может иногда привести к переполнению диска. Так же как и имя файла результатов, имя файла трассировки может включать строку \CMD{\%AUTODATE\%}.

В поле \CTL{Формат файлов} выбирается нужный формат всех создаваемых файлов. В настоящее время поддерживается два формата:

\begin{itemize}
\item \CTL{TXT} --- текстовый формат, в котором каждое измерение записывается в отдельной строке, внутри строки значения разделяются символом табуляции. Формат поддерживается большинством научных приложений.
\item \CTL{CSV} --- текстовый формат, в котором каждое измерение записывается в отдельной строке, внутри строки значения разделяются запятой. Формат поддерживается рядом офисных приложений, например Miscosoft Excel.
\end{itemize}

Флаг \CTL{Переписать файлы} указывает, нужно ли всякий раз в начале измерений переписывать файлы заново. Если флаг сброшен, все новые показания будут дописываться в конец файлов.

В поле \CTL{Комментарий} вводится произвольный фрагмент текста, который будет записан в начале файла результатов. Поле может быть пустым, но рекомендуется вводить в него краткое описание Образца и условия проведения измерений, это в будущем облегчит анализ результатов.
