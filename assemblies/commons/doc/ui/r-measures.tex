\label{sec_r_measure_config}

Как было описано в разделе <<Определение сопротивления>> (стр.~\pageref{sec_r_measures}), Установка может определять сопротивление Образца разными способами. В данном разделе указывается нужный способ и вводятся сопутствующие параметры.

Флаг \CTL{Вольтметром/ Амперметром} включает соответствующий метод измерения. Дополнительных параметров в данном случае нет, и напряжение и ток определяются автоматически.

Флаг \CTL{Вольтметром/ Вольтметром} включает соответствующий метод измерения. В данном режиме необходимо ввести номинал эталонного сопротивления и, желательно, его абсолютную погрешность.

Флаг \CTL{Вольтметром/ вручную} включает соответствующий метод измерения. В данном режиме необходимо явно ввести величину тока в цепи и её абсолютную погрешность. Величину тока можно узнать, например, из показаний ИП. Эта величина считается неизменной на всём протяжении эксперимента.

Флаг \CTL{Омметром} включает соответствующий метод измерения. Дополнительных параметров в данном случае нет, сопротивление определяется автоматически.

\IMPORTANT{При всяком изменении способа определения сопротивления Программа должна быть перезапущена, чтобы изменения вошли в силу.}