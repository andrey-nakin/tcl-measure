В поле \CTL{Тип термопары} выбирается один из поддерживаемых типов. Расшифровку типов термопар см. в таблице~\ref{tab_tc_types}

\begin{table*}
\begin{center}
\caption{Типы термопар}
\begin{tabular}{cl}
\hline \hline
B & Платина 30\% + Родий / Платина 60\% + Родий \\
C & Вольфрам 5\% + Рений / Вольфрам 26\% + Рений \\
E & Хромель / Константан \\
J & Железо / Константан \\
K & Хромель / Алюмель \\
N & Нихросил / Нисил \\
R & Платина 13\% + Родий / Платина \\
S & Платина 10\% + Родий / Платина \\
T & Медь / Константан \\
\hline \hline
\end{tabular}
\label{tab_tc_types}
\end{center}
\end{table*}

В поле \CTL{Опорная температура} вводится температура опорного спая термопары, которая считается постоянной на всём протяжении эксперимента. Например, если опорный спай погружён в жидкий азот, поле должно иметь значение $77.4$.

Флаг \CTL{Инв. полярность} меняет полярность подключения термопары. Если температура, отображаемая Программой, некорректная (слишком большая или слишком маленькая), это может быть вызвано тем, что термопара подключена с неправильной полярностью. Менять физическое подключение в этом случае не нужно, достаточно поменять значение флага на противоположное.

В поле \CTL{Выражение для коррекции} вводится арифметическое выражение, которое меняет некоторым образом температуру, полученную непосредственно от термопары. Как известно, всякая термопара требует калибровки, по результатам которой показания должны быть скорректированы для устранения систематической погрешности. В данном поле и вводится это выражение для коррекции.

Выражение включает в себя переменную, которая содержит в себе температуру, полученную непосредственно от термопары, одно или несколько чисел и арифметические операции, выражаемые символами \CMD{+}, \CMD{-}, \CMD{*} и \CMD{/}. Порядок операций~--- как принято в арифметике, для изменения порядка используются круглые скобки. Переменная обозначается произвольной латинской буквой. Между операндами допускается произвольное количество пробелов для улучшения читабельности.

{\bf Пример. } Пусть показания термопары дают температуру, увеличенную на $0.5$~К по сравнению с реальной, корректировка должна уменьшать исходное значение на $0.5$. Тогда выражение для корректировки будет \mbox{\CMD{x - 0.5}} (выражение вводится в поле без кавычек).

\bigskip

Для облегчения калибровки Программа может автоматически составить выражение для коррекции по двум контрольным измерениям температуры:

\begin{enumerate}
\item Сбросьте текущую калибровку (очистите поле \CTL{Выражение для коррекции}) и перезапустите Программу.
\item Выполните два контрольных измерения температуры, для этого погрузите рабочий спай термопары в среды с известной температурой (например, жидкий азот и тающий лёд). Значения температуры возьмите из поля \CTL{Температура} на вкладке \CTL{Измерение}.
\item Перейдите вновь на вкладку \CTL{Параметры установки} и нажмите кнопку \CTL{Калибровка}.
\item Введите значения контрольных температур и соответствующих им показаниям термопары, после чего нажмите кнопку \CTL{Расчёт}.
\end{enumerate}

Программа вычислит подходящее выражение для коррекции и запишет его в поле. После этого оператор может в случае необходимости вручную изменить это выражение.