Установка может самостоятельно вычислять и регистрировать удельное сопротивление Образца по измеренному абсолютному. Для этого перед началом измерений оператор должен ввести параметры Образца. В самом простом случае необходимо ввести только расстояние между потенциальными контактами $l$. Удельное сопротивление тогда вычисляется по формуле:

\begin{equation}
\rho = 2 \pi \cdot R \cdot l,
\end{equation}

\noindent где $\rho$~--- удельное сопротивление, $R$~--- абсолютное сопротивление между потенциальными контактами, $l$~--- расстояние между потенциальными контактами.

Для более точного определения удельного сопротивления необходимо знать поперечное сечение Образца. Для этого оператор должен кроме расстояния $l$ ввести ширину и толщину Образца. При этом предполагается, что Образец имеет форму параллелепипеда, а потенциальные контакты расположены вдоль направления тока. Удельное сопротивление тогда вычисляется по формуле:

\begin{equation}
\rho = \frac{R \cdot w \cdot t}{l},
\end{equation}

\noindent где $\rho$~--- удельное сопротивление, $R$~--- абсолютное сопротивление между потенциальными контактами, $l$~--- расстояние между потенциальными контактами, $w$~--- ширина Образца, перпендикулярная направлению тока, $t$~--- толщина Образца.
