В зависимости от способа измерения сопротивления измерительные приборы и источник питания подключаются следующим образом:

\subsubsection{Вольтметром/ амперметром}

    \begin{itemize}
        \item Выходы <<Input HI>> и <<Input LO>> мультиметра MM1 подключаются к выходам 1 и 2 разъёма J1. 
        \item Выходы <<Input I>> и <<Input LO>> мультиметра MM2 подключаются к выходам 3 и 4 разъёма J1. 
        \item Выходы источника питания I2 подключаются к выходам 1 и 2 разъёма J5. 
    \end{itemize}

\subsubsection{Вольтметром/ вольтметром}

    \begin{itemize}
        \item Выходы <<Input HI>> и <<Input LO>> мультиметра MM1 подключаются к выходам 1 и 2 разъёма J1. 
        \item Выходы <<Input HI>> и <<Input LO>> мультиметра MM2 подключаются к выходам 3 и 4 разъёма J1. 
        \item Выходы источника питания I2 подключаются к выходам 1 и 2 разъёма J5. 
        \item Тестовое сопротивление R3 подключается к выходам 3 и 4 разъёма J5. 
    \end{itemize}
        
\subsubsection{Вольтметром/ вручную}

    \begin{itemize}
        \item Выходы <<Input HI>> и <<Input LO>> мультиметра MM1 подключаются к выходам 1 и 2 разъёма J1. 
        \item Мультиметр MM2 не используется. 
        \item Выходы источника питания I2 подключаются к выходам 1 и 2 разъёма J5. 
    \end{itemize}
        
\subsubsection{Омметром}

    \begin{itemize}
        \item Выходы <<Input HI>> и <<Input LO>> мультиметра MM1 подключаются к выходам 3 и 4 разъёма J1. 
        \item Выходы <<Sense HI>> и <<Sense LO>> мультиметра MM1 подключаются к выходам 1 и 2 разъёма J1. 
        \item Мультиметр MM2 не используется. 
        \item Источник питания I2 не используется. 
    \end{itemize}
