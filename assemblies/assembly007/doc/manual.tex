\documentclass[12pt, a4paper, twocolumn]{report}
\usepackage[utf8]{inputenc}
\usepackage[russian]{babel}
\usepackage{hyperref}
\usepackage[]{graphicx}

%\makeindex

\input{../../commons/doc/style.tex}
\newcommand{\IMPORTANT}{{\bf ВНИМАНИЕ:~}}

\newcommand{\CTL}[1]{<<{\bf #1}>>}

\newcommand{\CMD}[1]{<<{\tt #1}>>}

\newcommand{\PARAM}[1]{\item {\bf #1} }

\newcommand{\PARAMSECTION}[1]{\vbox{}{\bf Раздел <<#1>>}}


\title{Установка №~7. \\ Регистрация удельного сопротивления в~зависимости от~температуры с~ручным управлением температурой образца. \\ Руководство пользователя}
\author{Накин~А.~В.}

\begin{document}

\maketitle

\tableofcontents

\chapter{Общие сведения}

Установка №~7 (далее~--- Установка) представляет собой программно-аппаратный комплекс для регистрации удельного сопротивления материалов (далее~--- Образцов) в зависимости от температуры.

Установка производит съём показаний измерительных приборов, вычисление значения сопротивления со всеми сопутствующими погрешностями и запись результатов в файлы данных.

Управление температурой Образца производится вручную при помощи:

\begin{itemize}
\item регулируемого трансформатора, питающего обмотку электропечи;
\item манипуляциями со сборкой, например погружением её в пары кипящего азота.
\end{itemize}

Установка сама никак не влияет на температуру Образца, а только регистрирует её вместе с сопротивлением.

Одновременно Установка способна работать только с одним единственным Образцом.

\section{Состав Установки}

\subsection{Состав аппаратной части}

Аппаратная часть Установки состоит из персональной ЭВМ (далее~--- ПЭВМ) и следующих приборов:

\begin{itemize}

\item Мультиметр 34410A/34401A (далее~--- мультиметр) компании Agilent (2--3~шт.). Предназначен для измерения сопротивления Образца (один или два мультиметра) и его температуры посредством измерения напряжения на термопаре. Мультиметры управляются ПЭВМ посредством USB или RS-232 интерфейса.

\item Произвольный источник питания (далее~--- ИП) постоянного тока (0--1~шт.). Предназначен для запитывания электрической цепи с Образцом. Может отсутствовать, если способ измерения сопротивления не требует запитки.

\item Блок из 8-ми управляемых 2-х позиционных реле МВУ-8 компании ОВЕН (1~шт.). Предназначен для коммутации приборов и Образца. Управляется ПЭВМ посредством прибора АС-4, к которому подключён посредством интерфейса RS-485.

\item Преобразователь интерфейсов USB/RS-485 АС-4 компании ОВЕН (1~шт.). Предназначен для подключения МВУ-8 к ПЭВМ через USB интерфейс.

\item Эталонное сопротивление (0--1~шт.). Предназначено для измерения тока в цепи посредством измерения падения напряжения на сопротивлении с известным номиналом. Может отсутствовать, если способ измерения сопротивления не требует эталонного сопротивления.

\end{itemize}

\subsection{Состав программной части}
\label{sec_software}

Программная часть Установки состоит из следующих компонентов.

\begin{itemize}

\item Программа Установки (далее~--- Программа), состоящая из нескольких модулей, написанных на языке Tcl.

\item Интерпретатор языка Tcl и необходимые библиотеки. Дополнительные сведения см. в соответствующих документах.

\item Библиотека, предоставляющая программный интерфейс VISA для доступа к приборам 34410A и E3645A. Например: программный пакет IO Libraries Suite компании Agilent. Дополнительные сведения см. в соответствующих документах.

\item Драйвера устройств, подключаемых непосредственно к ПК.


\end{itemize}

\section{Принцип работы}

Установка периодически с заданной частотой измеряет сопротивление и температуру Образца и записывает результаты в файлы, а также выводит на экран ПЭВМ для оперативного контроля. Оператор вручную управляет температурой Образца.

\label{sec_registration_types}

Частота регистрации (записи в файл) измерений вводится оператором перед началом измерений и может задаваться одним из следующих способов:

\begin{itemize}
\item {\bf Временная зависимость}~--- показания регистрируются с фиксированным временным интервалом, например один раз в секунду.
\item {\bf Температурная зависимость}~--- показания регистрируются с фиксированным температурным шагом, например $1$~Кельвин.
\item \label{sec_reg_type_manual} {\bf Вручную}~--- показания регистрируются по команде оператора.
\end{itemize}

\subsection{Определение температуры}

\label{sec_t_measures}

Температура Образца измеряется посредством термопары. Имеется два способа подключения термопары:

\subsubsection*{К вольтметру}

Термопара имеет два спая: рабочий и опорный. Рабочий спай помещается рядом с Образцом, а опорный спай погружается в термостабильную среду с известной температурой (например, жидкий азот). Важно, чтобы температура этой среды не менялась в течении всего эксперимента.

Свободные концы подключаются непосредственно к вольтметру, по показаниям которого определяется разность температур рабочего и опорного спаев.

Данный способ позволяет производить температурные измерения с достаточной точностью (до 0,1~К) и высокой скоростью (до 25~раз в секунду) без потери точности.

\subsubsection*{К ТРМ-201}

Термопара имеет один единственный рабочий спай и подключается к прибору ТРМ-201. Прибор сам определяет температуру <<холодного>> спая (свободных концов термопары), измеряет разность потенциалов на нём и преобразует её в абсолютное температурное значение.

Данный способ возволяет высвободить один мультиметр и не требует наличия термостабильной среды. Недостатки: большая погрешность определения температуры (до 0,5\%) и невысокая скорость измерения (не чаще одного отсчёта в секунду).

\subsection{Определение сопротивления}

\label{sec_r_measures}

Сопротивление Образца измеряется 4-х контактным методом, для чего к Образцу подводятся два потенциальных и два токовых контакта. Сопротивление определяется одним из следующих способов:

\subsubsection{Вольтметром/ амперметром}

Используются вольтметр, амперметр и ИП. Амперметр и ИП включены последовательно с Образцом, амперметр подключён к токовым контактам Образца. Вольтметр включён параллельно с Образцом и подключён к его потенциальным контактам. Сопротивление Образца определяется как $R = V/I$, где $R$~--- искомое сопротивление, $V$~--- показания вольтметра, $I$~--- показания амперметра.

Данный способ наиболее универсальный и точный, но требует максимального количества задействованных мультиметров.

\subsubsection{Вольтметром/ вольтметром}

Используются 2 вольтметра, ИП и эталонное сопротивление номинала $R_2$, включённое последовательно с Образцом и ИП. Первый вольтметр подключён  к потенциальным контактам Образца, второй вольметр --- к выходам эталонного сопротивления. Сопротивление Образца определяется как $R = R_2 V_1/V_2$, где $R$~--- искомое сопротивление, $V_1$~--- показания первого вольтметра, $V_2$~--- показания второго вольтметра.

Данный способ требует максимального количества задействованных мультиметров, а также высококачественное эталонное сопротивление. В ряде случаев он может обеспечить повышенную относительную точность измерений, когда сравниваются результаты измерений одного и того же Образца.

\subsubsection{Вольтметром/ вручную}
\label{sec_voltmeter_manually}

Используются вольтметр и ИП, Образец включён последовательно с ИП, вольтметр подключён к потенциальным контактам Образца. Ток в цепи $I$ считается известным  и неизменным в течении всего эксперимента. Сопротивление Образца определяется как $R = V/I$, где $R$~--- искомое сопротивление, $V$~--- показания вольтметра.

Данный способ наименее точный. Если ИП сам не производит измерение тока в цепи, требуется ручное измерение тока.

\subsubsection{Омметром}

Используется омметр, подключённый к Образцу 4-х контактным способом. Сопротивление Образца определяется непосредственным считыванием показаний омметра.

Данный способ наиболее простой, но не годится в том случае, когда сопротивление Образца слишком мало или велико, то есть выходит за диапазон точных измерений омметра.

\bigskip 

Во всех вышеприведённых способах функции вольметра, амперметра или омметра выполняет мультметр, переведённый в соответствующий режим.


\subsection{Удельное сопротивление}

Установка может самостоятельно вычислять и регистрировать удельное сопротивление Образца по измеренному абсолютному. Для этого перед началом измерений оператор должен ввести параметры Образца. В самом простом случае необходимо ввести только расстояние между потенциальными контактами $l$. Удельное сопротивление тогда вычисляется по формуле:

\begin{equation}
\rho = 2 \pi \cdot R \cdot l,
\end{equation}

\noindent где $\rho$~--- удельное сопротивление, $R$~--- абсолютное сопротивление между потенциальными контактами, $l$~--- расстояние между потенциальными контактами.

Для более точного определения удельного сопротивления необходимо знать поперечное сечение Образца. Для этого оператор должен кроме расстояния $l$ ввести ширину и толщину Образца. При этом предполагается, что Образец имеет форму парраллелипипеда, а потенциальные контакты расположены вдоль направления тока. Удельное сопротивление тогда вычисляется по формуле:

\begin{equation}
\rho = \frac{R \cdot w \cdot t}{l},
\end{equation}

\noindent где $\rho$~--- удельное сопротивление, $R$~--- абсолютное сопротивление между потенциальными контактами, $l$~--- расстояние между потенциальными контактами, $w$~--- ширина Образца, перпендикулярная направлению тока, $t$~--- толщина Образца.


\section{Принципиальная схема}

Принципиальная схема Установки приведена на рис.~\ref{pic-scheme}.

Разъёмы J1, J4, J5 расположены на передней панели соединительного бокса. Внутри бокса размещён блок реле МВУ-8, содержащий переключатели S1--S4.

Образец R1 подключён к выходам разъёма J4: токовые контакты к выходам 2 и 6, потенциальные --- к выходам 3 и 7.

Источник питания постоянного тока I1, если используется, подключается к выходам 1, 2 разъёма J5.

Эталонное сопротивление R2, если используется, подключается к выходам 3, 4 разъёма J5.

Мультиметр MM1 может работать:

\begin{enumerate}
\item как вольтметр, измеряя напряжение на потенциальных контактах Образца;
\item либо как омметр, измеряя полное сопротивление Образца; в этом случае мультиметр MM2 не используется, а к выходам 3, 4 разъёма J1 подключены выходы <<Input HI>> и <<Input LO>> мультиметра MM1.
\end{enumerate}

Мультиметр MM2 может работать:

\begin{enumerate}
\item как амперметр, измеряя ток в цепи Образца;
\item как вольтметр, измеряя падение напряжения на эталонном сопротивлении R2;
\item либо отсутствовать.
\end{enumerate}

Термопара подключена к выходам 1 и 5 разъёма J4. Напряжение на ней измеряется мультиметром MM3, работающим всегда в режиме вольтметра и подключённого к выходам 5, 6 разъёма J1.

Полярность подключения мультиметров, ИП и термопары не важна.

Сдвоенный переключатель S1 предназначен для изменения полярности подключения вольтметра к потенциальным контактам Образца. Сдвоенный переключатель S2 предназначен для изменения полярности подключения ИП в цепь Образца. В данной Установке переключение полярностей не производится.

Переключатель S3 предназначен для размыкания цепи Образца. Размыкание производится:

\begin{itemize}
\item либо в момент переключения полярности посредством S1/S2;
\item либо при измерении сопротивления при помощи омметра, поскольку в данном случае запитка Образца не требуется.
\end{itemize}

Переключатель S4 используется в режиме <<Вольтметром/вручную>> (стр.~\pageref{sec_voltmeter_manually}), когда амперметр MM2 отсутствует: в этом случае S4 переводится на нижнее положение и замыкает цепь питания Образца.

На схеме не изображено подключение блока реле МВУ-8 к внешнему питанию и сети RS-485.

\begin{figure*}
\begin{center}
\includegraphics[width=1.0\textwidth, clip, viewport=190 120 630 530]{scheme}
\end{center}
\caption{Принципиальная схема Установки}
\label{pic-scheme}
\end{figure*}

\chapter{Подготовка к работе}

\section{Подготовка аппаратной части}

\subsection{Коммутация мультиметров и ИП}

Соединения внутри соединительного бокса обеспечиваются шлейфом №~6 (далее~--- шлейф).

В зависимости от способа измерения сопротивления измерительные приборы и источник питания подключаются следующим образом:

\subsubsection{Вольтметр + Амперметр}

    \begin{itemize}
        \item Выходы <<Input HI>> и <<Input LO>> мультиметра MM1 подключаются к выходам 1 и 2 разъёма J1. 
        \item Выходы <<Input I>> и <<Input LO>> мультиметра MM2 подключаются к выходам 3 и 4 разъёма J1. 
        \item Выходы внешнего источника питания I2 подключаются к выходам 1 и 2 разъёма J5. 
    \end{itemize}

\subsubsection{Вольтметр + Вольтметр}

    \begin{itemize}
        \item Выходы <<Input HI>> и <<Input LO>> мультиметра MM1 подключаются к выходам 1 и 2 разъёма J1. 
        \item Выходы <<Input HI>> и <<Input LO>> мультиметра MM2 подключаются к выходам 3 и 4 разъёма J1. 
        \item Выходы внешнего источника питания I2 подключаются к выходам 1 и 2 разъёма J5. 
        \item Тестовое сопротивление R3 подключается к выходам 3 и 4 разъёма J5. 
    \end{itemize}
        
\subsubsection{Вольтметр + ручное измерение тока}

    \begin{itemize}
        \item Выходы <<Input HI>> и <<Input LO>> мультиметра MM1 подключаются к выходам 1 и 2 разъёма J1. 
        \item Мультиметр MM2 не используется. 
        \item Выходы внешнего источника питания I2 подключаются к выходам 1 и 2 разъёма J5. 
    \end{itemize}
        
\subsubsection{Омметр}

    \begin{itemize}
        \item Выходы <<Input HI>> и <<Input LO>> мультиметра MM1 подключаются к выходам 1 и 2 разъёма J1. 
        \item Выходы <<Sense HI>> и <<Sense LO>> мультиметра MM1 подключаются к выходам 3 и 4 разъёма J1. 
        \item Мультиметр MM2 не используется. 
        \item Внешний источник питания I2 не используется. 
    \end{itemize}


\subsection{Подготовка Образца и термопары}

Подключите Образец, термопару и, возможно, эталонное сопротивление к разъёмам соединительного бокса согласно принципиальной схеме.

\subsection{Подготовка источника питания}

\begin{enumerate}

\item Включите прибор. Убедитесь, что он подключён к ПЭВМ.
\item Проверьте подключение измерительных проводов: оно должно соответствовать схеме Установки и режиму измерения.

\end{enumerate}


\subsection{Подготовка мультиметров 34410A/34401A}

\begin{enumerate}

\item Включите прибор. Убедитесь, что он подключён к ПЭВМ.
\item Проверьте подключение измерительных проводов: оно должно соответствовать схеме Установки и режиму измерения.

\end{enumerate}


\subsection{Подготовка МВУ-8}

\begin{enumerate}

\item Включите прибор. Убедитесь, что он подключён к ПЭВМ.
\item Проверьте подключение измерительных проводов: оно должно соответствовать схеме Установки и режиму измерения.

\end{enumerate}


\subsection{Подготовка АС-4}

\begin{enumerate}

\item Включите прибор. Убедитесь, что он подключён к ПЭВМ.
\item Проверьте подключение измерительных проводов: оно должно соответствовать схеме Установки и режиму измерения.

\end{enumerate}


\section{Подготовка программной части}

Запустите программу Установки №~7.

Убедитесь в работоспособности Программы и всех аппаратных устройств. Для этого убедитесь, что Программа определяет и выводит на экран ПЭВМ температуру и сопротивления Образца. Откройте вкладку \CTL{Измерение} (она открыта сразу после запуска Программы). В текстовых полях (\CTL{Ток}, \CTL{Напряжение} и т.~д.), а также на графиках должны выводиться обработанные показания приборов. На графике производной температуры по времени $dT/dt$ показания могут выводиться с небольшой задержкой.

После того, как показания приборов начали отображаться на вкладке, убедитесь, что они лежат в ожидаемом диапазоне значений, что говорит о правильности подключения всех устройств и работы Установки. Если некоторые показания явно некорректные, проверьте качество соединений, положение кнопок <<Front/Rear>> мультиметров и пр.

\subsection{Параметры Образца}

Откройте закладку \CTL{Образец}. Данная вкладка содержит параметры измеряемого образца.

\subsubsection{Геометрические параметры}

Если возможно, введите некоторые геометрические параметры. Они будут использоваться для вычисления удельного сопротивления Образца. 

В поле \CTL{Расстояние между потенциальными контактами} введите соответствующее расстояние и его погрешность. В самом простом случае этого достаточно для вычисления удельного сопротивления.

Если образец имеет форму близкую к параллелипипеду, то введите соответствующие размеры в полях \CTL{Длина}, \CTL{Ширина} и \CTL{Толщина}, а также их абсолютные погрешности. При этом длина Образца не используется в расчётах, а ширина и толщина используются для расчёта поперечного сечения.

Допускается, хотя и не рекомендуется, не указывать погрешности геометрических параметров. В этом случае соответствующие поля ввода должны быть пустыми.

Если геометрические параметры Образца неизвестны, очистите все соответствующие поля. В этом случае удельное сопротивление не будет вычисляться автоматически, но на общую работу установки это не повлияет.


\subsubsection{Файлы}

В поле \CTL{Имя файла результатов} вводится путь и имя файла, в который будут записаны результаты измерений для последующего использования. Как правило, это поле обязательно для заполнения. Если поле пустое, то в процессе измерений их результаты не будут фиксироваться в файле, что как правило недопустимо.

Имя файла может включать в себя строчку \CMD{\%AUTODATE\%} (без кавычек). Тогда путь к реальному файлу будет в этом месте включать три
 подкаталога, соответствующих текущим году, месяцу и дню месяца. Например, если в поле введено \CMD{C:\textbackslash{}res\textbackslash{}\%AUTODATE\%\textbackslash{}res.txt}, и измерения проводятся 5 января 2012 года, то реальный путь к файлу будет \CMD{C:\textbackslash{}res\textbackslash{}2012\textbackslash{}01\textbackslash{}05\textbackslash{}res.txt}.

В поле \CTL{Имя файла трассировки} вводится путь и имя файла, в который будут записаны необработанные результаты всех произведённых измерений. Как правило, трассировка нужна только на этапе отладки Установки. Размер файла трассировки значительно превышает размер файла результатов, что может иногда привести к переполнению диска. Так же как и имя файла результатов, имя файла трассировки может включать строку \CMD{\%AUTODATE\%}.

В поле \CTL{Формат файлов} выбирается нужный формат всех создаваемых файлов. В настоящее время поддерживается два формата:

\begin{itemize}
\item \CTL{TXT} --- текстовый формат, в котором каждое измерение записывается в отдельной строке, внутри строки значения разделяются символом табуляции. Формат поддерживается большинством научных приложений.
\item \CTL{CSV} --- текстовый формат, в котором каждое измерение записывается в отдельной строке, внутри строки значения разделяются запятой. Формат поддерживается рядом офисных приложений, например Miscosoft Excel.
\end{itemize}

Флаг \CTL{Переписать файлы} указывает, нужно ли всякий раз в начале измерений переписывать файлы заново. Если флаг сброшен, все новые показания будут дописываться в конец файлов.

В поле \CTL{Комментарий} вводится произвольный фрагмент текста, который будет записан в начале файла результатов. Поле может быть пустым, но рекомендуется вводить в него краткое описание Образца и условия проведения измерений, это в будущем облегчит анализ результатов.


\subsection{Параметры измерения}

Откройте вкладку <<Параметры измерения>>. Содержимое данной вкладки определяет условия проведения измерений.

\subsubsection{Метод регистрации}

В данном разделе выбирается способ и частота регистрации измерений (см. описание принципа работы на стр.~\pageref{sec_registration_types}).

Если требуется регистрация с фиксированным временным интервалом, то выберите флаг \CTL{Временная зависимость}, после чего в поле \CTL{Временной шаг} введите значение интервала.

Если требуется регистрация с фиксированным температурным интервалом, то выберите флаг \CTL{Температурная зависимость}, после чего в поле \CTL{Температурный шаг} введите значение интервала.

Если требуется нерегулярная регистрация, то выберите флаг \CTL{Вручную}.

\subsubsection{Метод измерения сопротивления}

\label{sec_r_measures}

Сопротивление Образца измеряется 4-х контактным методом, для чего к Образцу подводятся два потенциальных и два токовых контакта. Сопротивление определяется одним из следующих способов:

\subsubsection{Вольтметром/ амперметром}

Используются вольтметр, амперметр и ИП. Амперметр и ИП включены последовательно с Образцом, амперметр подключён к токовым контактам Образца. Вольтметр включён параллельно с Образцом и подключён к его потенциальным контактам. Сопротивление Образца определяется как $R = V/I$, где $R$~--- искомое сопротивление, $V$~--- показания вольтметра, $I$~--- показания амперметра.

Данный способ наиболее универсальный и точный, но требует максимального количества задействованных мультиметров.

\subsubsection{Вольтметром/ вольтметром}

Используются 2 вольтметра, ИП и эталонное сопротивление номинала $R_2$, включённое последовательно с Образцом и ИП. Первый вольтметр подключён  к потенциальным контактам Образца, второй вольметр --- к выходам эталонного сопротивления. Сопротивление Образца определяется как $R = R_2 V_1/V_2$, где $R$~--- искомое сопротивление, $V_1$~--- показания первого вольтметра, $V_2$~--- показания второго вольтметра.

Данный способ требует максимального количества задействованных мультиметров, а также высококачественное эталонное сопротивление. В ряде случаев он может обеспечить повышенную относительную точность измерений, когда сравниваются результаты измерений одного и того же Образца.

\subsubsection{Вольтметром/ вручную}
\label{sec_voltmeter_manually}

Используются вольтметр и ИП, Образец включён последовательно с ИП, вольтметр подключён к потенциальным контактам Образца. Ток в цепи $I$ считается известным  и неизменным в течении всего эксперимента. Сопротивление Образца определяется как $R = V/I$, где $R$~--- искомое сопротивление, $V$~--- показания вольтметра.

Данный способ наименее точный. Если ИП сам не производит измерение тока в цепи, требуется ручное измерение тока.

\subsubsection{Омметром}

Используется омметр, подключённый к Образцу 4-х контактным способом. Сопротивление Образца определяется непосредственным считыванием показаний омметра.

Данный способ наиболее простой, но не годится в том случае, когда сопротивление Образца слишком мало или велико, то есть выходит за диапазон точных измерений омметра.

\bigskip 

Во всех вышеприведённых способах функции вольметра, амперметра или омметра выполняет мультметр, переведённый в соответствующий режим.


\chapter{Измерения}

После подготовки к работе аппаратной и программной частей Установки можно приступить к измерениям.

Выберите вкладку \CTL{Измерение}. Ещё раз убедитесь в том, что все приборы работают, величины температуры и сопротивления Образца измеряются и находятся в ожидаемом диапазоне значений.

После этого нажмите кнопку \CTL{Начать запись}. Установка начнёт регистрацию температуры и сопротивления.

Оператор управляет температурой образца, например увеличением напряжения на обмотке электропечи. Скорость изменения температуры определяется требованиями эксперимента.

Если выбран режим ручной регистрации (стр.~\pageref{sec_reg_type_manual}), то оператор должен самостоятльно нажимать кнопку \CTL{Снять точку} всякий раз, когда требуется зафиксировать измерение.

Кнопка \CTL{Снять точку} доступна и в других режимах, при регистрации временной и температурной зависимостей. То есть даже при автоматической регистрации оператор может вручную зарегистрировать нужное измерение. Например, если выбранный временной интервал довольно велик, а оператор наблюдает <<интересное поведение>> Образца, то он может вручную зафиксировать текущее измерение, даже если временной интервал ещё не истёк.

По окончании измерений нажмите кнопку \CTL{Остановить запись}, после чего программа зафиксирует все результаты в файлах. Файлы далее доступны для анализа.

По окончании работы нажмите кнопку \CTL{Выход}. Программа произведёт сброс всех устройств в исходное состояние и закончит работу.

%\section{Подготовка программной части}
%
%Подготовка программной части сводится к установке всех необходимых программных компонентов (см. раздел <<\hyperref[sec_software]{Состав программной части}>>) и настройке программы Установки.
%
%Установка программных компонентов рассмотрена в соответствующих документах.
%
%Для настройки параметров программы Установки (далее~--- Программы) необходимо её запустить.
%
%\subsection{Запуск Программы}
%
%Для запуска Программы наберите команду следующего вида:
%
%{\small
%\begin{verbatim}
% wish85 -encoding utf-8 assembly004.tcl
%\end{verbatim}
%}
%
%Здесь {\tt wish85}~--- название программы-интерпретатора Tcl версии 8.5 (версия, а значит и название может отличаться); {\tt assembly004.tcl}~--- имя главного модуля Программы.
%
%Рекомендуется настроить операционную систему таким образом, чтобы запуск любой программы, написанной на Tcl и имеющей расширение {\tt .tcl} можно было осуществлять просто кликом по ней из менеджера файлов.
%
%Если все необходимые программные компоненты присутствуют и правильно настроены, откроется окно Программы.
%
%\subsection{Настройка параметров программы}
%
%Все параметры Программы условно делятся на две части:
%
%\begin{itemize}
%
%\item \emph{Параметры установки}: параметры, которые не требуют частого изменения, такие как адреса устройств. Как правило, эти параметры, будучи настроенными, не требуют изменения до тех пор, пока не изменится конфигурация устройств Установки.
%
%\item \emph{Параметры измерения}: параметры, влияющие на процесс измерения. Как правило, эти параметры требуют подстройки перед каждым сеансом работы Установки.
%
%\end{itemize}
%
%\subsubsection{Параметры установки}
%
%Ниже приведены параметры установки. Все поля обязательны для заполнения, если не оговорено обратное.
%
%\begin{itemize}
%
%\PARAM{Порт для АС-4}
%
%COM-порт, направленный на нужный экземпляр АС-4. 
%
%\PARAM{Сетевой адрес МВУ-8}
%
%МВУ-8, как и всякое устройство в сети RS-485, имеет уникальный внути сегмента сети 8-ми битный адрес, представленный целым десятичным числом. Как правило, он написан на самом устройстве МВУ-8. Чтобы узнать и, при необходимости, изменить адрес запустите программу <<Конфигуратор МВУ-8>>.
%
%\PARAM{VISA адрес источника питания}
%
%VISA-адрес ИП E3645A. Если в Установке используется ручное управление питанием, можно оставить это поле пустым.
%
%\PARAM{VISA адрес вольтметра на образце}
%
%VISA-адрес мультиметра 34410A, работающего в режиме вольметра и измеряющего напряжение на Образце.
%
%\PARAM{VISA адрес вольтметра на эталоне}
%
%VISA-адрес мультиметра 34410A, работающего в режиме вольметра и измеряющего напряжение на эталонном сопротивлении (если оно используется в Установке), или же в режиме амперметра и измеряющего непосредственно ток в цепи.
%
%\PARAM{Звуковой сигнал по окончании}
%
%Если данный флажок отмечен, по окончании измерений один из мультиметров подаст короткий звуковой сигнал.
%
%\end{itemize}
%
%\subsubsection{Параметры измерения}
%
%Ниже приведены параметры измерения. В зависимости от конфигурации Установки часть полей может быть <<запрещена>>, то есть недоступна для редактирования. Все остальные поля обязательны для заполнения, если не оговорено обратное.
%
%\PARAMSECTION{Питание образца}
%
%\begin{itemize}
%
%\PARAM{Ручное управление}
%
%Если данный флажок отмечен, Установка работает в режиме ручного управления питанием. При этом поля для ввода диапазона изменения тока запрещены.
%
%\PARAM{Начальный ток}
%
%В автоматизированном режиме управления питанием это поле определяет начальный ток, то есть ток, при котором будет произведено первое измерение.
%
%\PARAM{Конечный ток}
%
%В автоматизированном режиме управления питанием это поле определяет конечный ток, то есть ток, при котором будет произведено последнее измерение. Значение должно быть больше или равно значению начального тока.
%
%\PARAM{Приращение}
%
%В автоматизированном режиме управления питанием это поле определяет изменение тока на каждом шаге.
%
%\PARAMSECTION{Параметры измерения}
%
%\PARAM{Циклов 50 Гц на измерение}
%
%Для уменьшения погрешности мультиметры производят многократные измерения в течении некоторого времени, после чего возвращают усреднённое значение в качестве результата. Поскольку основным источником шумов является переменный ток сети питания, периоды измерения привязаны к периодам колебаний тока в сети. В документации к мультиметру Agilent 34410A этот параметр называется NPLC (Number of Power Line Cycles~--- число циклов линии питания). В наших условиях частота сети равна 50~Гц, следовательно каждый цикл имеет продолжительность 20~мс.
%
%Для того, чтобы свести эффект влияния шумов  линии питания к минимум, данный параметр должен иметь целое значение. Рекомендуемое значение параметра~--- 10, в этом случае каждое измерение производится в течении 200~мс.
%
%\PARAM{Измерений на точку}
%
%Установка может производить многократные измерения при одном и том же токе, после чего вычислять и сохранять среднее значение в качестве результата. Разброс значений будет учитываться при вычислении погрешности измерений.
%
%Минимальное значение параметра~--- 1. Если позволяют условия эксперимента, рекомендуется производить не менее 100 измерений.
%
%Время измерения зависит от величины параметров <<Циклов 50 Гц на измерение>> и <<Измерений на точку>>. Например, если первый параметр имеет значение 10, а второй~--- 100, то каждая точка результирующей зависимости будет измеряться в течении $0,02 \cdot 10 \cdot 100 = 20$~секунд (здесь $0,02$~--- продолжительность одного периода колебаний напряжения в сети). При использовании переполюсовок продолжительность измерения увеличивается в 2 (при переполюсовке напряжения \emph{или} тока) или в 4 (при переполюсовке и напряжения и тока) раза.
%
%\PARAM{Игнорировать инстр. погрешность}
%
%В некоторых случаях может потребоваться исключить из результирующей ошибки измерения сопротивления инструментальную погрешность. В этом случае флажок должен быть отмечен.
%
%При отключённой инструментальной погрешности ошибкой измерения является стандартное отклонение величины сопротивления в результате нескольких измерений (количество которых определяет параметром <<Измерений на точку>>).
%
%\PARAMSECTION{Метод измерения тока}
%
%\PARAM{Амперметром}
%
%Если флажок отмечен, сила тока измеряется непосредственно амперметром.
%
%\PARAM{Напряжением на эталоне}
%
%Если флажок отмечен, сила тока измеряется через падение напряжение на эталонном сопротивлении.
%
%\PARAM{Эталонное сопротивление}
%
%В данном поле вводится величина эталонного сопротивления, которое используется для измерения силы тока в цепи.
%
%\PARAMSECTION{Переполюсовки}
%
%\PARAM{Переполюсовка напряжения}
%
%Если флажок отмечен, Установка в процессе измерения производит переключение полярности подключения вольтметра к Образцу.
%
%\PARAM{Переполюсовка тока}
%
%Если флажок отмечен, Установка в процессе измерения производит переключение полярности подключения ИП.
%
%\PARAMSECTION{Файл результатов}
%
%\PARAM{Имя файла}
%
%Имя файла, куда будут сохраняться результаты измерения. Если поле пусто, запись в файл не производится.
%
%\PARAM{Формат файла}
%
%Формат результирующего файла. Может иметь следующие значения:
%
%\begin{itemize}
%\item TXT~--- каждое измерение занимает одну строку, значения в строке разделены символом табуляции (09h). Комментарием считается строка, начинающаяся с символа <<\#>>.
%\item CSV~--- каждое измерение занимает одну строку, значения в строке разделены запятой. Первая строка является комментарием. Файлы данного формата удобны для открытия в программе MS~Excel и аналогах.
%\end{itemize}
%
%\PARAM{Переписать файл}
%
%Если флажок отмечен, в начале измерений предыдущее содержимое файла уничтожается. Такой режим удобен при автоматизированном управлении током в цепи.
%
%Если флажок сброшен, результаты будут добавляться в конец файла без его перезаписи. Такой режим удобен при ручном управлении питанием.
%
%\end{itemize}

\end{document}
